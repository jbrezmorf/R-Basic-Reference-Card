\documentclass[8pt,landscape]{article}
%\usepackage{pslatex}
\usepackage{graphicx}
\usepackage{multicol}
\usepackage{calc}
\usepackage{color}
\usepackage{wrapfig}

\usepackage{hyperref}
\usepackage[]{url}
\hypersetup{%
  colorlinks=true,  
  linkcolor = black,
  citecolor = black, 
  urlcolor = blue, 
  pdfpagemode=UseNone, 
  pdfstartview=FitH,
  pdfauthor= {Jonas Stein, Tom Short, Emanuel Paradis},
  pdftitle= R Reference Card
}

\urlstyle{same}


% Turn off header and footer
\pagestyle{empty}

%\setlength{\leftmargin}{0.75in}
\setlength{\oddsidemargin}{-0.75in}
\setlength{\evensidemargin}{-0.75in}
\setlength{\textwidth}{10.5in}


\setlength{\topmargin}{-0.3in}
\setlength{\textheight}{7.4in}
\setlength{\headheight}{0in}
\setlength{\headsep}{0in}

\pdfpageheight\paperheight
\pdfpagewidth\paperwidth

% Redefine section commands to use less space
\makeatletter
\renewcommand\section{\@startsection{section}{1}{0mm}%
                                     {-24pt}% \@plus -12pt \@minus -6pt}%
                                     {0.5ex}%
                                {\color{blue}\normalfont\large\bfseries}}
\makeatother

% Don't print section numbers
\setcounter{secnumdepth}{0}

\setlength{\parindent}{0pt}
\setlength{\parskip}{0pt}

\newcommand{\code}{\texttt}
\newcommand{\bcode}[1]{\texttt{\textbf{#1}}}
\newcommand\F{\code{FALSE}}
\newcommand\T{\code{TRUE}}

%\newcommand{\hangpara}[2]{\hangindent#1\hangafter#2\noindent}
%\newenvironment{hangparas}[2]{\setlength{\parindent}{\z@}\everypar={\hangpara{#1}{#2}}}

\newcommand{\describe}[1]{\begin{description}{#1}\end{description}}


% -----------------------------------------------------------------------

\begin{document}

%\raggedright
\footnotesize
\begin{multicols*}{3}

% multicol parameters
% These lengths are set only within the two main columns
%\setlength{\columnseprule}{0.25pt}
\setlength{\premulticols}{1pt}
\setlength{\postmulticols}{1pt}
\setlength{\multicolsep}{1pt}
\setlength{\columnsep}{2pt}

\begin{center}
     {\Large\textbf{\color{blue}R Basic Reference Card}}
\end{center}
%
%Examples in this document use the variables \code{df} = data frame object,
%\code{v} = vector, \code{s} = string,  \code{f} = filename as string all others are not yet rewritten.
%
%\section{Getting help}
\vspace{-2ex}
\bcode{?topic}  documentation on \code{topic} 

\bcode{help.search("topic")} search the help system 

\bcode{apropos("topic")} all objects matching the reg. exp. "topic"

\bcode{ls()} list all objects; \code{pat="pat"} to search on a pattern


\vspace{-7ex}
\section{Input and output}
\everypar={\hangindent=9mm}

\bcode{source("my.R")} includes and executes my.R in this place

\bcode{data(f)} loads specified data sets

\bcode{install.package()}

\bcode{library(s)} load add-on packages

\bcode{read.table(f)} reads a file into data frame; \code{sep=""} for value separator; \code{header=TRUE} names on the first line; \code{as.is=TRUE} 
                prevent string to factor conversion

\bcode{read.csv(f,header=TRUE)} same,  for comma-delimited files

\bcode{read.delim(f",header=TRUE)} same, for tab-delimited files

\bcode{read.fwf(f,widths,header=FALSE,sep="\t",as.is=FALSE)} for \emph{f}ixed \emph{w}idth \emph{f}ormatted data 


\bcode{save(f,...)} saves the specified objects (...) in the XDR format

\bcode{save.image(f)} saves all objects

\bcode{load(f)} load the datasets written with \code{save}

\bcode{print(a, ...)} prints its arguments; \emph{generic} function

\bcode{format(x,...)} format an R object for pretty printing

\bcode{write.table(x, file=f,row.names=TRUE, col.names=TRUE, sep=" ")} 
convert \code{x} to data frame and output to \code{f}; see params: \code{quote}, \code{sep}, \code{eol}, \code{na}, \code{col.names=NA}

\bcode{sink(f)} output to \code{f}, until \code{sink()}

\everypar={\hangindent=0mm}
Exchange tables with other apps. (Excel) via clipboard: \\
\code{df <- read.table("clipboard")}\\
\code{write.table(df,"clipboard",sep="\textbackslash t",col.names=NA)}


\section{Data creation}
\everypar={\hangindent=9mm}

\bcode{c(...)} combine arguments to vector, \emph{generic}, see param \code{recursive}


\bcode{from:to} generates a sequence;\code{2:5} returns \code{[1] 2 3 4 5}

\bcode{seq(from,to)} generates a sequence, \code{by=} set step; \code{length=} set length

\bcode{seq(along=x)} generates \code{1, 2, ..., length(x)}; useful for loops

\bcode{rep(x,n)} replicate \code{x} \code{n}-times (abcabc); \code{each=} to get (aabbcc)

\bcode{data.frame(...)} data frame from the list of vector parameters; 
  \code{data.frame(v=1:6,ch=c("a","B"))}; recycle shorter vectors

\bcode{list(...)} create a list of the named or unnamed arguments;
  \code{list(a=c(1,2),b="hi",c=3i)}; 

\bcode{array(x,dim=)} array with data \code{x}; specify
dimensions like \code{dim=c(3,4,2)}; elements of \code{x} recycle if \code{x}
is not long enough

\bcode{matrix(x,nrow=,ncol=)} matrix; elements of \code{x} recycle

\bcode{outer(x,y, FUN(x,y) )} create matrix by tensor product

\bcode{factor(x,levels=)} encodes a vector \code{x} as a factor

\bcode{gl(n,k,length=n*k,labels=1:n)} factor with \code{n} labels $k$-times each

\bcode{expand.grid()} data frame of all combinations of given lists

\bcode{cbind(df1, df2), rbind(df1,df2)} combine arguments by columns/rows for matrix-like objects

\section{Object operations} 
\everypar={\hangindent=9mm}

\bcode{rm(myvar)} removes object \code{myvar} from memory

\bcode{rm(list = ls(all = TRUE))} removes all objects from memory

\bcode{class(X)} class of object \code{X} (get type of data)

\bcode{as.array(x), as.data.frame(x), as.numeric(x), ...} \\
convert type;  \code{methods(as)} shows a complete list
    
\bcode{is.na(x), is.null(x), is.array(x), is.numeric(x), ...}\\
test for type; \code{methods(is)}

\bcode{summary(a)} gives a ``summary'' of \code{a},  \emph{generic} function

\bcode{length(x)}  number of elements in \code{x}

\bcode{dim(x)} get or set the dimension of an object;
\code{dim(x) <- c(3,2)}

\bcode{dimnames(x)} get or set the dimension names of an object

\bcode{nrow(x)}, \bcode{ncol(x)} number of rows/cols of matrix (cf. \code{NROW}, \code{NCOL})

%\bcode{class(x)} get or set the class of \code{x}; \code{class(x) <- "myclass"}

%\bcode{unclass(x)} remove the class attribute of \code{x}

%\bcode{attr(x,which)} get or set the attribute \code{which} of \code{x}

%\bcode{attributes(obj)} get or set the list of attributes of \code{obj}



\section{addressing vectors}
\begin{tabular}{@{}l@{\ \ \ }l}
\code{v[n]} & \code{n}$^{th}$ element\\
\code{v[-n]} & all \emph{but} the \code{n}$^{th}$ element\\
\code{v[1:n]} & first \code{n} elements\\
\code{v[-(1:n)]} & elements from \code{n+1} to the end\\
\code{v[c(1,4,2)]} & specific elements\\
\code{v["name"]} & element named \code{"name"}\\
\code{v[v > 3 \& v < 5]} & all elements between 3 and 5\\
\code{v[v \%in\% set]} & elements in vector \code{set}
\end{tabular}

\vspace{-6ex}
\section{addressing lists}
\begin{tabular}{@{}l@{\ \ \ }l}
\code{x[n]} or \code{x[[n]]}& \code{n}$^{th}$ element of the list\\
\code{x[["name"]]} or \code{x\$name} & element of the list named \code{"name"}
\end{tabular}

\vspace{-6ex}
\section{addressing matrices and dataframes}
\begin{tabular}{@{}l@{\ \ \ }l}
\code{x[i,j]} & element at row \code{i}, column \code{j}\\
\code{x[i,]} & row \code{i}\\
\code{x[,j]} & column \code{j}\\
\code{x[,c(1,3)]} & columns 1 and 3\\
\code{x["name",]} & row named \code{"name"}\\
{\bf only for dataframes:} & \\
\code{df[["name"]]} or \code{df\$name}& column named \code{"name"}
\end{tabular}

\vspace{-6ex}




\section{Data selection and manipulation}
\everypar={\hangindent=9mm}

\bcode{which.max(v), which.min(v)}  index of the max/min element of
\bcode{v}

\bcode{rev(v)}  reverses the elements of \code{v}

\bcode{sort(v)}  sorts \code{v} (increasing); use \code{rev(sort(x))} for decreasing

\bcode{cut(x,breaks)}  divides \code{x} into intervals (laels of resulting factor)\\ \code{breaks} is number of intervals or vector of cut points

\bcode{match(x, y)}  for every \code{x[i]} index of same value in \code{y}, \code{NA} otherwise

\bcode{which(x)}  indices of TRUE in logical vector \code{x}; e.g. \code{which(x>7)}

\bcode{na.omit(x)}  suppress \code{NA} values (or lines of matrix or data frame)

\bcode{na.fail(x)}  returns an error message if \code{x} contains at least one \code{NA}

\bcode{unique(x)}  suppress duplicate values (or lines of matrix or DF)

\bcode{table(x)}  freqency/{\bf contingency table} for vector/data frame

\bcode{subset(df, x)}  lines where \code{x} is TRUE; e.g. \code{subset(df, V1 < 5)}

\bcode{sample(x, N)}  random sample of size \code{N} from vector \code{x};\code{replace=FALSE}

\bcode{prop.table(x,margin=)} table entries as fraction of marginal table 




\section{Math}
\everypar={\hangindent=9mm}

\bcode{sin,cos,tan,asin,acos,atan,atan2,log,log10,exp}

\bcode{range(x)}  shortcut for \code{c(min(x), max(x))}

\bcode{sum(x)}  sum of the elements of \code{x}

\bcode{diff(x)}  iterated differences of vector \code{x}

\bcode{prod(x)}  product of the elements of \code{x}

\bcode{mean(x)}  mean of the elements of \code{x}

\bcode{median(x)}  median of the elements of \code{x}

\bcode{quantile(x,probs=)} sample quantiles; \code{type=} nine types of approx.

\bcode{weighted.mean(x, w)} mean of \code{x} with weights \code{w}

\bcode{rank(x)}  ranks of the elements of \code{x}

\bcode{var(x)}  variance of the elements of \code{x};  

\bcode{sd(x)} standard deviation of \code{x}

\bcode{cov(x)}  covariance matrix of the matrix or data frame\code{x}

\code{cor(x)}  correlation matrix of \code{x} see param. \code{method=}

\bcode{var(x, y)} or \code{cov(x, y)}  covariance between \code{x} and \code{y} 

\bcode{cor(x, y)}  linear correlation between \code{x} and \code{y} (also for matrices)

\bcode{choose(n, k)}  $k$-combinations from $n$ elements = $n!/[(n-k)!k!]$

\bcode{round(x, n)}  rounds the elements of \code{x} to \code{n}
decimals

\bcode{scale(x)}  normalization of vector (matrix, df) $x$

\bcode{pmin(x,y,...)}, \bcode{pmax(x,y,...)}  parallel minimum, maximum

\bcode{cumsum(v)} vector with $i^{th}$ element \code{sum(v[:i])}

\bcode{cumprod(v)}, \bcode{cummin(v)}, \bcode{cummax(v)} \dots similar

%\bcode{union(x,y)},~\bcode{intersect(x,y)},~\bcode{setdiff(x,y)},~\bcode{setequal(x,y)}, \bcode{is.element(el,set)} ``set'' functions

%\bcode{fft(v)} Fast Fourier Transform; \code{inverse=T} unnormalized inverse

%\bcode{convolve(x,y)} convolutions of two vectors (via FFT); see \code{type=}
     
%\bcode{filter(x,filter)} applies linear filtering to a univariate time series

\everypar={\hangindent=0mm}
Many math functions have a logical parameter \code{na.rm=FALSE} to
specify missing data (NA) removal.




\section{Matrices}
\everypar={\hangindent=9mm}

\bcode{t(x)} transpose

\bcode{diag(x)} diagonal

\bcode{\%*\%} matrix multiplication and scalar product

\bcode{solve(a,b)} solves \code{a \%*\% x = b} for \code{x}

\bcode{solve(a)} matrix inverse of \code{a}

\bcode{rowSums(x)}, \bcode{colSums(x)}, \bcode{rowMeans}, \bcode{colMeans} \\
sum/mean of rows/cols for a matrix-like object

\section{Advanced data processing}
\everypar={\hangindent=9mm}

\bcode{function( arglist ) expr} function definition; \bcode{return(value)}

\bcode{lapply(X,FUN)} apply \code{FUN} to each element of the list \code{X}

%\bcode{apply(X,MARGIN,FUN,...)} apply \code{FUN} with named arguments \code{...} to slices prependicular to dimension(s) \code{MARGIN} of array \code{X}

%\bcode{tapply(X,INDEX,FUN, ...)} apply \code{FUN} to each cell
%of a ragged array given by \code{X} with indexes \code{INDEX}

\bcode{by(data,FACTOR,FUN,...)} split data frame \code{data} by \code{FACTOR} of same length and apply \code{FUN} to 
resulting data frames

\bcode{merge(a,b)} merge two data frames (default by common columns)

\bcode{ftable(xtabs(cols~rows,data=x))} a contingency table from DF

%\bcode{aggregate(df,by,FUN)} splits \code{df} into subsets, computes summary statistics for
%     each, and returns the result in a convenient form; \code{by} is a
%     list of grouping elements, each as long as the variables in \code{df}

\bcode{stack, unstack(x,...)} convert list of factor vectors to/from DF


%\bcode{reshape(x, ...)} reshapes a data frame between 'wide' format with
%     repeated measurements in separate columns of the same record and
%     'long' format with the repeated measurements in separate records;
%     use \code(direction="wide") or \code(direction="long")

%\bcode{do.call(funname, args)} executes a function call from the name of
%  the function and a list of arguments to be passed to it


\section{Distributions}

\bcode{rnorm(n, mean=0, sd=1)} Gaussian (normal)  

\bcode{rlnorm(n, meanlog=0, sdlog=1)} lognormal  

\bcode{rweibull(n, shape, scale=1)} Weibull

\bcode{rgamma(n, shape, scale=1)} gamma  

\bcode{rbeta(n, shape1, shape2)} beta

\bcode{rt(n, df)} `Student' ($t$)  

\bcode{rf(n, df1, df2)} Fisher--Snedecor ($F$)

\bcode{rchisq(n, df)} Pearson   ($\chi^2$)  

\bcode{rexp(n, rate=1)} exponential

\bcode{rpois(n, lambda)} Poisson

\bcode{rcauchy(n, location=0, scale=1)} Cauchy  

\bcode{rbinom(n, size, prob)} binomial  

\bcode{rhyper(nn, m, n, k)} $nn$-white drown, $m$-white, $n$-black, $k$-drown

\bcode{rgeom(n, prob)} geometric  

\bcode{rnbinom(n, size, prob)} negative binomial  

\bcode{rlogis(n, location=0, scale=1)} logistic  

\bcode{runif(n, min=0, max=1)} uniform  

\bcode{rwilcox(nn, m, n)}, \code{rsignrank(nn, n)} Wilcoxon's statistics  

\bcode{d<distr>(x, ...)} density function

\bcode{p<distr>(x, ...)} distribution (CDF)

\bcode{q<distr>(p, ...)} quantile function ($0 \le p \le 1$)




\section{Tests and confidence intervals}
\everypar={\hangindent=9mm}
\bcode{t.test()} Student's test

\bcode{pairwise.t.test()} ...corrections for ANOVA posthoc analysis

\bcode{power.t.test()} power of $t$-test

\bcode{binom.test()} exact test for $p$

\bcode{var.test()} F-test for variance of normal distribution

\bcode{fisher.test()} exact test for independence in contingency table

\bcode{ks.test()} one or two sample Kolmogorov-Smirnov test

\bcode{kruskal.test(x)} Kruskal-Wallis (``robust'' ANOVA for list \code{x})

\bcode{shapiro.test()} test of normality

\bcode{apropos("test")} for full list 

\section{Statistics}
\bcode{approx(x,y=)} piecewise linear or constant interpolation

\bcode{spline(x,y=)} cubic spline interpolation



\bcode{lm(formula, data=, subset=)} fit linear model; \code{formula} is typically of
     the form \code{response ~ termA + termB + ...}; use \code{I(x*y)
     + I(x\^{}2)} for terms made of nonlinear components

\bcode{aov(formula)} analysis of variance model

{\color{blue}\bf formulas}

\bcode{x+c:d}  with interaction term $y=\beta_0 + \beta_x x + \beta_{cd}$;  

\bcode{a-1} without intercept

\bcode{c*d} same as \code{c+d+c:d} same as \code{(c+d)\^2}

\bcode{I(x\^2)} $y=\beta_0 + \beta_2 x^2$
     
%\bcode{glm(formula,family=)} fit generalized linear models, specified by
%     giving a symbolic description of the linear predictor and a
%     description of the error distribution; \code{family} is a
%     description of the error distribution and link function to
%          be used in the model; see \code{?family}



%\bcode{loess(formula)} fit a polynomial surface using local fitting
{\color{blue}\bf work with \code{fit}}

\everypar={\hangindent=9mm}
\bcode{predict(fit,...)}  predictions from \code{fit} based on input data

\bcode{df.residual(fit)}  number of residual degrees of freedom

\bcode{coef(fit)}  returns the estimated coefficients

\bcode{residuals(fit)}  returns the residuals

\bcode{deviance(fit)}  returns the deviance

\bcode{fitted(fit)}  returns the fitted values

%\bcode{logLik(fit)}  computes the logarithm of the likelihood and the number of parameters

%\bcode{AIC(fit)}  computes the Akaike information criterion or AIC


\bcode{anova(fit,...)} ANOVA table

\bcode{plot(fit)} 



\vspace{-6ex}
\section{Basic Plotting}

\everypar={\hangindent=9mm}

\bcode{plot(y)}  plot of the values of \code{y} (on the $y$-axis) ordered on the $x$-axis

\bcode{plot(vx, vy)}  scatter plot; points (\code{vx[i]}, \code{vy[i]})

\bcode{matplot(x,y)}  $i^{th}$ col of matrix $x$ vs. $i^{th}$ col of $y$ taking $i^{th}$ value from vector plot params (col, bg, pch,..)

\bcode{stripchart(x)}  plot of the values of \code{x} on a line 

\bcode{hist(x)}  histogram of the frequencies of \code{x}

\bcode{barplot(x)}  histogram of the values of \code{x}; horizontal: \code{horiz=FALSE}

\bcode{pie(x)}  circular pie-chart

\bcode{boxplot(x)}  ``box-and-whiskers'' plot

\bcode{pairs(x)}  plot for every pair of columns in data frame  \code{x}

\bcode{pairs({\textasciitilde{}}V1+V2+V3,data=df)} only for columns 1,2,3

\bcode{plot.ts(x)}  plot time series object \code{x}

\bcode{ts.plot(x)}  multivariate series may have different dates

\bcode{qqnorm(x)}  quantiles of \code{x} with respect to \code{qnorm()}

\bcode{qqplot(x, y)}  quantiles of \code{y} with respect to the quantiles of \code{x}

\bcode{contour(x, y, z)}  \code{x}, \code{y} - vectors; \code{z}$=z(x,y)$

\bcode{filled.contour(x, y, z)}  same + fill between contours

\bcode{image(x, y, z)}  plot $z$ as colour

\bcode{persp(x, y, z)}  3D graph with perspective 

\bcode{dotchart(x)}  Cleveland's plot, cf. dotchart2 from package Hmisc

\bcode{stars(x)}  draw star for every row of df \code{x}; \code{draw.segments=}

\bcode{coplot(x\~{}y $\mid$ z)}  conditioning scatter plots for each value or interval of values of \code{z}

\bcode{symbols(x, y, ...)}  draw parametic symbols (termometer, circle, ...) at \code{x},\code{y} points



\section{Statistic Plotting}


\bcode{fourfoldplot(x)}  plot 2 by 2 by k contingency table $x$visualizes, with quarters of circles, the association between two dichotomous variables for different populations (\code{x} must be an array with \code{dim=c(2, 2, k)}, or a matrix with \code{dim=c(2, 2)} if $k=1$)

\bcode{assocplot(x)}  Cohen--Friendly graph showing the deviations  from independence of rows and columns in a two dimensional contingency table

\bcode{mosaicplot(x)}  `mosaic' graph of the residuals from a log-linear regression of a contingency table

\bcode{interaction.plot (f1, f2, y)}  plot means of \code{y} ($y$-axis) for factors \code{f1} ($x$-axis) and \code{f2} (line type) 


\bcode{termplot(mod.obj)}  plot (partial) effects of a regression model

\vspace{-5ex}
\section{Low-level plotting commands}
\everypar={\hangindent=9mm}

\bcode{points(x, y)}  adds points (the option \code{type=} can be used)

\bcode{lines(x, y)}  id. but with lines

\bcode{text(x, y, labels, ...)}  add  text \code{labels} at points (\code{x}, \code{y})

\bcode{mtext(text, side=3, ...)}  add \code{text} at margin specified by \code{side}

\bcode{segments(x0, y0, x1, y1)}  lines from (\code{x0},\code{y0}) to (\code{x1},\code{y1})

\bcode{arrows(x0, y0, x1, y1, angle= 30, code=2)}  same with arrows 

\bcode{abline(a,b)}  draws a line of slope \code{b} and intercept \code{a}

\bcode{abline(h=y)}  draws a horizontal line at ordinate \code{y}

\bcode{abline(v=x)}  draws a vertical line at abcissa \code{x}

\bcode{abline(lm.obj)}  draws the regression line given by \code{lm.obj}

\bcode{rect(x1, y1, x2, y2)}  rectangle with corners (\code{x1}, \code{y2}) (\code{x2}, \code{y2})

\bcode{polygon(x, y)}  polygon linking the points (\code{x}, \code{y})

\bcode{legend(x, y, legend)}  add legend (\code{x},\code{y})

\bcode{title()}  adds a title and optionally a sub-title

\bcode{axis(side, ...)}  add axis on side (1=below, 2=left, 3=above, 4=right)

\bcode{rug(v)}  draws the data on the $x$-axis as small vertical lines

\bcode{locator(N, type=``n'', ...)}  returns coordinates ($x,y$) after the user has clicked \code{n} times on the plot; also draws symbols \code{type}



\section{common plot params}

\bcode{add=FALSE}  if \T{} superposes the plot on the previous one (if it exists)

\bcode{axes=TRUE}  if \F{} does not draw the axes and the box

\bcode{type="p"}  specifies the type of plot, \code{"p"}: points, \code{"l"}: lines, \code{"b"}: points connected by lines, \code{"o"}: id. but the lines are over the points, \code{"h"}: vertical lines, \code{"s"}: steps, the data are represented by the top of the vertical lines, \code{"S"}: id. but the data are represented by the bottom of the vertical lines

\bcode{xlim=, ylim=}  specifies the lower and upper limits of the axes, for example with \code{xlim=c(1, 10)} or \code{xlim=range(x)}

\bcode{xlab=, ylab=}  annotates the axes, must be variables of mode character

\bcode{main=}  main title, must be a variable of mode character

\bcode{sub=}  sub-title (written in a smaller font)







\vspace{-5ex}
\section{Graphical parameters}
\everypar={\hangindent=7mm}

\bcode{par(...)} set grapics parameters globally

\bcode{adj}  adjustement of text (0=left, 0.5=center, 1=right)

\bcode{bg}  background color (fill); \code{colors()} list color names

\bcode{bty} box type around plot, (\code{"o"}, \code{"l"}, \code{"7"}, \code{"c"}, \code{"u"}, \code{"]"}) or \code{bty="n"} for none

\bcode{cex}  scale size of texts and symbols (also: \code{cex.axis}, ...)

\bcode{col}  color of symbols and lines; use color names or \code{"\#RRGGBB"};
see \code{rgb()}, \code{hsv()}, \code{gray()}, and \code{rainbow()};

\bcode{font}  style of text (\code{1}: normal, \code{2}: italics, \code{3}: bold, \code{4}: bold italics)

\hbox to 0.5\linewidth {%
    \vbox{%    
        \bcode{las}  orientation of axis labels \\
            \code{0}: parallel, \code{1}: horizontal,\\
            \code{2}: perpendicular, \code{3}: vertical
            
        \bcode{lty}  line type: 
    }
    \kern-4cm\includegraphics[width=4cm]{./lines.pdf}
}

\bcode{lwd}  line width

\bcode{mfrow=c(nr,nc)} set matrix \code{nr} $\times$ \code{nc} of subplots

\bcode{pch}  point type (integer code) or single character
and 25, or any single character within \code{""}

\samepage\includegraphics[width=8.5cm]{pch_symbol} 

\bcode{ps}   size of texts

\bcode{pty}  plotting region type, \code{"s"}: square, \code{"m"}: maximal

\bcode{xpd}  if set to \code{TRUE}, do not clip objects poking out 
%\bcode{tck}  tick's length as fraction of min of plot height and width

%\bcode{tcl}  tick's length as  fraction of text height (default \code{tcl=-0.5})

%\bcode{xaxt}  if \code{xaxt="n"} the $x$-axis is set but not drawn (useful in conjonction with \code{axis(side=1, ...)})

%\bcode{yaxt}  if \code{yaxt="n"} the $y$-axis is set but not drawn (useful in conjonction with \code{axis(side=2, ...)})


\section{Lattice (Trellis) graphics}
\everypar={\hangindent=9mm}

\bcode{xyplot(y\~{}x)}  bivariate plots (with many functionalities)

\bcode{barchart(y\~{}x)}  histogram of the values of \code{y} with
respect to those of \code{x}

\bcode{dotplot(y\~{}x)}  Cleveland dot plot (stacked plots line-by-line
and column-by-column)

\bcode{densityplot(\~{}x)}  density functions plot

\bcode{histogram(\~{}x)}  histogram of the frequencies of \code{x}

\bcode{bwplot(y\~{}x)}  ``box-and-whiskers'' plot

\bcode{qqmath(\~{}x)}  quantiles of \code{x} with respect to the values expected under a theoretical distribution

\bcode{stripplot(y\~{}x)}  single dimension plot, \code{x} must be numeric, \code{y} may be a factor

\bcode{qq(y\~{}x)}  quantiles to compare two distributions, \code{x} must be numeric, \code{y} may be numeric, character, or factor but must have two `levels'

\bcode{splom(\~{}x)}  matrix of bivariate plots

\bcode{parallel(\~{}x)}  parallel coordinates plot

\bcode{levelplot(z\~{}x*y|g1*g2)}  coloured plot of the values of \code{z} at the coordinates given by \code{x} and \code{y} (\code{x}, \code{y} and \code{z} are all of the same length)

\bcode{wireframe(z\~{}x*y|g1*g2)}  3d surface plot

\bcode{cloud(z\~{}x*y|g1*g2)}  3d scatter plot

%\everypar={\hangindent=0mm}
%In the normal Lattice formula, \code{y~x|g1*g2} has
%combinations of optional conditioning variables \code{g1}
%and \code{g2} plotted on separate panels. Lattice functions 
%take many of the same arguments as base
%graphics plus also \code{data=} the data frame for the formula
%variables and \code{subset=} for subsetting. Use \code{panel=} to
%define a custom panel function (see \code{apropos("panel")}
%and \code{?llines}). Lattice functions return an object of class
%trellis and have to be \code{print}-ed to produce the graph. Use
%\code{print(xyplot(...))} inside functions where automatic
%printing doesn't work. Use \code{lattice.theme} and \code{lset} to
%change Lattice defaults.










\end{multicols*}
\end{document}
